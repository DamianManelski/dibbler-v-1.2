%%
%% $Id: dibbler-user-faq.tex,v 1.11 2007-09-06 23:10:40 thomson Exp $
%%

\newpage
\section{Frequently Asked Questions}

Soon after initial Dibbler version was released, feedback from user
regarding various things started to appear. Some of the questions were
common enough to get into this section.

\subsection{Common Questions}

%% Q1 
\Q Why client does not configure routing after assigning addresses, so
I cannot e.g. ping other hosts?

\A It's a common misunderstanding. DHCPv4 provides many configuration
parameters to host, with default router address being one of
them. Things are done differently in IPv6. Routing configuration is
supposed to be conveyed using Router Advertisements (RA) messages,
announced periodically by routers. Hosts are supposed to listen to
those messages and configure their routing appropriately. Note that
this mechanism is completely separate from DHCPv6. It may sound a bit
strange, but that's the way it was meant to work. 

Properly implemented clients are supposed to configure leased address
with /128 prefix and learn the actual prefix from RA. As this is
incovenient, many clients (with dibbler included) bend the rules and
configure received addresses with /64 prefix. Please note that this
value is arbitrary chosen and may be improper in many scenarios. 

\Note This behaviour has changed in the 0.5.0 release. Previous
releases configured received address with /128 prefix. To restore old,
more RFC conformant behavior, see \emph{strict-rfc-no-routing}
directive in the \ref{client-cfg-file} section.

\vspace{0.5cm}
%% Q2
\Q Dibbler server receives SOLICIT message, prints information about
ADVERTISE/REPLY transmission, but nothing is actually transmitted. Is
this a bug?

\A Are you sure that your client is behaving properly and responds to
Neighbor Discovery (ND) requests? Before any IPv6 packet (that
includes DHCPv6 message) is transmitted, recipient reachabity is
checked (using Neighbor Discovery protocol \cite{rfc4861}). Server
sends Neighbor Solicititation message and waits for client's Neighbor
Advertisement. If that is not transmitted, even after 3 retries,
server gives up and doesn't transmit IPv6 packet (DHCPv6 reply, that
is) at all. Being not able to respond to the Neighbor Discovery
packets may indicate invalid client behavior.

%% Q3
\Q Dibbler sends some options which have values not recognized by the
Ethereal/Wireshark or by other implementations. What's wrong?

\A DHCPv6 is a relatively new protocol and additional options are in a
specification phase. It means that until standarisation process is
over, they do not have any officially assigned numbers. Once
standarization process is over (and RFC document is released), this
option gets an official number. 

There's pretty good chance that different implementors may choose
diffrent values for those not-yet officialy accepted options. To
change those values in Dibbler, you have to modify file
misc/DHCPConst.h and recompile server or client. See Developer's
Guide, section \emph{Option Values} for details.

%% Q4
\Q I can't get (insert your trouble causing feature here) to
work. What's wrong? 

\A Go to the project \url{http://klub.com.pl/dhcpv6/}{homepage} and
browse \href{http://klub.com.pl/lists/dibbler/}{list archives}. If
your problem was not reported before, please don't hesitate to write
to the
\href{http://klub.com.pl/cgi-bin/mailman/listinfo/dibbler}{mailing
  list} or \href{mailto:thomson(at)klub.com.pl}{contact author}
directly. 

\subsection{Linux specific questions}

%% Q1
\Q I can't run client and server on the same host. What's wrong?

\A First of all, running client and server on the same host is just
plain meaningless, except testing purposes only. There is a problem
with sockets binding. To work around this problem, consult Developer's
Guide, Tip section how to compile Dibbler with certain options.

%% Q2
\Q After enabling unicast communication, my client fails to send
REQUEST messages. What's wrong?

\A This is a problem with certain kernels. My limited test capabilites
allowed me to conclude that there's problem with 2.4.20
kernel. Everything works fine with 2.6.0 with USAGI patches. Patched 
kernels with enhanced IPv6 support can be downloaded from
\url{http://www.linux-ipv6.org/}. Please let me know if your kernel
works or not.

\subsection{Windows specific questions}

%% Q1
\Q After installing \emph{Advanced Networking Pack} or \emph{Windows XP
  ServicePack2} my DHCPv6 (or other IPv6 application) stopped
working. Is Dibbler compatible with Windows XP SP2?

\A Both products (Advanced Networking Pack as well as Service Pack 2
for Windows XP) provide IPv6 firewall. It is configured by default to
reject all incoming IPv6 traffic. You have to disable this
firewall. To disable firewall on the ,,Local Area Connection''
interface, issue following command in a console:

\begin{lstlisting}
netsh firewall set adapter "Local Area Connection" filter=disable
\end{lstlisting}

%% Q2
\Q Server or client refuses to create DUID. What's wrong?

\A Make sure that you have at least one up and running interface with
at least 6 bytes long MAC address. Simple ethernet or WIFI card
matches those requirements. Note that network cable must be plugged
(or in case of wifi card -- associated with access point), otherwise
interface is marked as down.

%% Q3
\Q Is Microsoft Windows 8 supported?

\A Unfortunately, Windows 8 is not supported yet.
