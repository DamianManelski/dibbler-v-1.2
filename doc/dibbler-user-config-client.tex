\newpage
\section{Client configuration}

This section describes Dibbler server, relay  and client
configuration. Square brackets denotes optional values: mandatory
[optional]. Alternative is marked as $\mid$. A $\mid$ B means A or
B. Parsers are case-insensitive, so Iface, IfAcE, iface and IFACE mean
the same. This does not apply to interface names. eth0 and
ETH0 are dwo diffrent interfaces.

\subsection{Data types}
Config file parsing is token-based. Token can be considered a keyword
or a specific phrase. Here are more commonly used types:
\begin{description}
\item[IPv6 address] -- IPv6 address, e.g. 2000:db8:1::dead:beef
\item[32-bit decimal integer] -- string containing only numbers, e.g. 12345
\item[string] -- string of arbitrary characters enclosed in single or double
  quotes, e.g. 'this is a string'. If string contains only a-z, A-Z and
  0-9 characters, quotes can be omited, e.g. beeblebrox
\item[DUID identifier] -- hex number starting with 0x,
  e.g. 0x12abcd. In Dibbler version 0.8.0RC1, another format was
  introduced: 2 hex digits separated by colon, e.g. 12:aa:bb:cc:d5. As
  this format may in some cases be confused with IPv6 address, the old
  format (starting with 0x) remains to be supported.
\item[IPv6 address list] -- IPv6 addresses separated with commas,
           e.g. 2001:db8:1::face:b00c, fe80::abcd:1234,::1
\item[DUID list] -- DUIDs separated with commas, e.g. 0x0123456,0x0789abcd
\item[string list] -- strings separated with comas, e.g. tealc,jackson,carter,oneill
\item[boolean] -- YES, NO, TRUE, FALSE, 0 or 1. Each of them can be
  used, when user is expected to enable or disable specific option.
\end{description}

\subsection{Scopes}
\label{scope}
There are four scopes, in which options can be specified: global,
inteface, IA and address. Every option is specific for one scope.
Each option is only applied to a scope and all subscopes in which it is
defined. For example, T1 is defined for
IA scope. If there are several interfaces and each has several address
classes, repeating the same T1 value many time may be a bit awkward.
Therefore parameters can be also used in more common scopes. In this
case -- in interface or global. Defining T1 in interface scope means:
,,for this interface the default T1 value is ...''. The same applies
to global scope. Options can be used multiple times. In that case
value defined later is used.

Global scope is the largest. It covers the whole config file and
applies to all intefaces, IAs, and addresses, unless some lower scope
options override it. Next scope is inteface. Options defined there
are inteface-specific and apply to a given interface, all IAs in this
interface and addresses in those IAs. Next is IA scope. Options
defined there are IA-specific and apply to this IA and to addresses it
contains. The narrowest scope is address or prefix.

\subsubsection{Interface declaration}
\label{client-scope-iface}
Each system interface, which should be configured, must be mentioned in
the configureation file. Interfaces can be declared with this syntax:
\begin{lstlisting}
iface interface-name
{
  interface-options
  IA-options
  address-options
}
\end{lstlisting}

or

\begin{lstlisting}
iface interface-number
{
  interface-options
  IA-options
  address-options
}
\end{lstlisting}

In the latter case, interface-number denotes interface index (or
ifindex). It can be obtained from \verb+ip~l+ (Linux), \verb+ipv6~if+
(Windows) or sometime \verb+ifconfig+ on other systems.
\verb+interface-name+ is an interface name.  Name of the
interface does not have to be enclosed in single or double quotes. It
is necessary only in Windows systems, where interface names sometimes
contain spaces, e.g. ''local network connection''.  Interface scoped
options can be used here. IA-scoped as well as address scoped options
can also be used. They will be treated as a default values for future
definitions of the IA and address instantations.

\subsubsection{IA declaration}
\label{client-scope-ia}
IA is an acronym for Identity Association. It is a logical entity
representing address or addresses used to perform some functions. It
is not suitable for prefixes (see Section \ref{pd-declaration}).
IA-options can be defined, e.g. T1. IPv6 addresses can be defined
here. All those values will be used as hints for a server.  Almost
always, each DHCPv6 client will have exactly one IA on each
interface. IA is declared using following syntax:

\begin{lstlisting}
ia [iaid]
{
  IA-options
  address-options
  address-declaration
}
\end{lstlisting}

IAID is an optional number, which describes identifier of given IA. If
not specified, it will be automatically assigned integer numbers
starting with 1. There may be more than one IA defined on an
interface. IA and PD (see Section \ref{pd-declaration}) may be used
together.

\subsubsection{TA declaration}
\label{client-scope-ta}
TA (Temporary Address) is a mechanism very similar to IA that allows
configuration of temporary addresses. See Section \ref{client-scope-ia}.

\subsubsection{PD declaration}
\label{pd-declaration}
\label{client-scope-pd}
PD (Prefix Delegation) is a mechanism that allows leasing prefixes in
similar way as addresses. For details, see \cite{rfc3633}. PD in
client config file causes client to send IA\_PD option. This option
informs server that client is requesting prefix delegation.

\begin{lstlisting}
pd [iaid]
{
  pd-options
  prefix-declaration
}
\end{lstlisting}
IAID is an optional number, which identifies this particular PD. If
not specified, it will be automatically assigned integer numbers
starting with 1. There may be more than one PD defined on an
interface. IA (see Section \ref{client-scope-ia}) may be used together.

\subsubsection{Address declaration}
\label{client-scope-addr}
When IA is defined, it is sometimes useful to define its address. Its
value will be used as a hint for the server. Address is declared in the
following way:

\begin{lstlisting}
address [number]
{
  address-options
  address-declaration
}
\end{lstlisting}
where number is optional and denotes how many addresses with those values should be
requested. If it is diffrent than 1, then IPv6 address options are not
allowed. Only address scoped options can be used here. Address can be
defined only within IA scope.

\subsubsection{Prefix declaration}
\label{client-scope-prefix}
When PD is defined, it is sometimes useful to define its prefix. Its
value will be used as a hint for the server. Prefix is declared in the
following way:

\begin{lstlisting}
prefix [number]
{
  prefix-options
  prefix
}
\end{lstlisting}

\subsection{Stateless configuration}

If interface does not contain \verb+IA+ or \verb+TA+ keywords, client
will ask for one address (one IA with one address request will be sent).
If client should not request any addresses on this interface,
\opt{stateless}\footnote{In the version 0.2.1-RC1 and earlier, this
  directive was called no-ia. This depreciated name is valid for now,
  but might be removed in future releases.} keyword must be used. In
such circumstances, only specified options will be requested.

\subsection{Relay support}
Usage of the relays is not visible from the client's point of view:
Client can't detect if it communicates via relay(s) or directly
with the server. Therefore no special directives on the client side
are required to use relays. See section \ref{feature-relays} for
details related to relay deployment.

\subsection{Comments}

Comments are allowed in configuration files. All common comment styles are supported:
\begin{itemize}
\item C++ style one-line comments: \verb+// this is comment+
\item C style multi-line comments: \verb+ /* this is multi-line comment */+
\item script style one-line comments: \verb+# this is one-line comment+
\end{itemize}

\subsection{File location}
\label{client-cfg-file}
Client configuration file should be named \verb+client.conf+. It
should be placed in the \verb+/etc/dibbler/+ directory (Linux system)
or in the current directory (Windows systems). One of design
requirements for client was ,,out of the box'' usage. To achieve this,
simply use empty \verb+client.conf+ file. Client will try to get one
address for each up and running interface. More preciselt client tries
to configure each up, multicast-capable and running interface, which
has link address at least 6 bytes long. This usually means that client
running in auto-detection mode will not configure tunnels, which
usually have IPv4 address (4 bytes long) as their link address. It
should configure all wired (Ethernet) and wireless (802.11)
interfaces, though.

\subsection{Client Reference}
\label{client-conf-reference}
This section contains complete list of parameters that are allowed in
client configuration file.

Dibbler client supports multiple parameters. Some ore defined by the
base DHCPv6 specification (a RFC 3315 document \cite{rfc3315}),
e.g. IA address option. Other parameters are for definition of one of
multitude of extensions that were defined for DHCPv6 protocol
(e.g. prefix delegation option). Finally, there are many configuration
parameters that are not options in DHCPv6 sense, but rather affect the
way the software operates (e.g. log level). All those parameters may
be defined in client config file. Every statement has defined
scope. See Section~\ref{scope} for details. In many cases, parameters
may also be defined in scopes larger than its default scope. For
example, instead of configuring DNS server option on 3 interfaces, it
can be defined once in global scope.

\begin{description}
\item[iface] -- (scope: global). Takes one parameter that can be
either string (interface name) or integer (interface index). Defines
that client should perform some actions on specified interface. Exact
operations are defined within interface scope. Can optionally take
second string parameter with the only allowed value of
``no-config''. If it is present, it instructs client to not perform
any operation on said interface. See Section \ref{client-scope-iface}.

\item[ia] -- (scope: interface). Defines IA\_NA (Identity Association for
Non-temporary Addresses), often abbreviated as IA. That is a
container for ``regular'' (non-temporary in DHCPv6 nomenclature)
addresses. Simply saying, this is a client's request for a single normal
address. There may be more than one ia defined on one interface. In
such case, client will request several addresses. It may have one
optional integer parameter that defines unique indentifier (IAID). If
followed by curly brackets, it will create new IA scope. See
Section \ref{client-scope-ia}.

\item[ta] -- (scope: interface). Defines IA\_TA (Indentity Association
for Temporary Addresses), often abbreviated as TA. This is a
container for temporary addresses. Simply saying, this is a client's
request for a temporary address. If followed by curly brackets, it
will create new IA scope, similar to IA. See
Section \ref{client-scope-ia}. Note that TA scope accepts only limited set of
parameters (e.g. iaid).

\item[pd] -- (scope: interface). Defines IA\_PD (Identity Association
for Prefix Delegation), often abbreviated as PD. This is a container
for prefixes. Simply saying, this is client's request for prefix
delegation. It may have one optional integer parameter that defines
unique indentifier (IAID). If followed by curly brackets, it will
create new PD scope. See Sections \ref{client-scope-pd}
and \ref{feature-prefix}.

\item[address] -- (scope: ia or ta). Defines an IPv6 address. It is
usually defined in IA or TA to specify that an address should be sent. It
takes one optional parameter that defines multiplier. For example, to
define 3 addresses, following syntax may be used:
\begin{lstlisting}
address 3 { }
\end{lstlisting}
If followed by curly brackets, creates new address scope. See
Section \ref{client-scope-addr}.

\item[prefix] -- (scope: pd). Defined an IPv6 prefix. It is defined in
PD to specify that a prefix should be sent. May be empty
(\verb+prefix+) or accompanied with prefix definition that consists of
IPv6 address followed by slash and prefix length (e.g.
\verb+prefix 2001:db8:1::/56+. If followed by curly brackets, creates
new prefix scope. See Section \ref{client-scope-prefix}.

\item[work-dir] -- (scope: global). Takes one string
 parameter. Defines working directory.

\item[downlink-prefix-ifaces] -- (scope: global). Takes coma separated
list of network interfaces. When client receives prefix from upstream
router, it attempts to split it into remaining interfaces. It works in
most cases, but if there are strange interfaces or specific
requirements, this auto-selection mechanism can be disabled and list
of downlink interfaces can be explicitly specified. This command is
used for that purpose. See \ref{feature-prefix} and
\ref{example-client-prefix}.

\item[log-level] -- (scope: global). Takes one integer
parameter. Defines verbose level of the log messages. The valid range
is from 1 (very quiet) to 8 (very verbose). Those values are modelled
after levels used in syslog. These are: 1(Emergency), 2(Alert),
3(Critical), 4(Error), 5 (Warning), 6(Notice), 7(Info) and
8(Debug). Currently Dibbler is using levels 3 to 8, as 1 and 2 are
reserved for system wide emergency events.

\item[log-name] -- (scope: global). Takes one string
parameter. Defines than name, which will be used during logging.

\item[log-mode] -- (scope: global). Takes one parameter that can be
   short, full, precise or syslog. Defines logging mode. In the
   default, full mode, name, date and time in the h:m:s format will be
   printed. In short mode, only minutes and seconds will be printed
   (this mode is useful on terminals with limited width). Precise mode
   logs information with seconds and microsecond precision. It is a
   useful as a performance diagnostic tool for finding bottlenecks in
   the DHCPv6 autoconfiguration process. Syslog works under POSIX
   systems (Linux, Mac OS X, BSD family) and allows default POSIX
   logging functions.

\item[log-colors] -- (scope: global). Takes one boolean parameter.
  Defines if logs printed to console should use colors. That feature
  is used to enhance logs readability.  As it makes the log files
  messy on systems that do not support colors, it is disabled by
  default. The default is off.

 \item[strict-rfc-no-routing] -- (scope: global) Takes one boolean
  parameter. The default value is 0. During normal operation, DHCPv6
  client should add IPv6 address only (i.e. configure it with /128
  prefix), without configuring routing. Routing is expected to be
  configured with Router Advertisements \cite{rfc4861}. Please see
  \href{http://klub.com.pl/bugzilla3/show_bug.cgi?id=222}{discussion
  in bug 222} for detailed discussion about that behavior. Note that
  Dibbler versions between 0.5.0RC1 and 1.0.0RC1 used to configure
  addressed with arbitrarily chosen (guessed) prefix length of /64.
  Although it was convenient for users, as in most cases the guess
  was correct and clients connected to the same link could ping each
  other immediately, its correct operation was based on the assumption
  that the guess is correct. If it isn't, tricky to debug problems
  will appear. Hosts will incorrectly assume that some off-the link
  hosts are on link (or vice versa) and will attempt to reach them
  directly. If you really understand the repercussions and still
  willing to use that old behavior, you can
  use \emph{strict-rfc-no-routing 0}. Author recommends against
  that, though.

\item[obey-ra-bits] -- (scope: global). Rounter Advertisements contain
  two bits that inform what kind of DHCPv6 services are available on
  link. \b M (Managed) that tells that addresses and prefixes can be
  obtained using stateful DHCPv6. \b O (OtherConf) tells that other
  configuration options may be configured. Both bits are defined
  in \cite{rfc4861}, section 4.2. It should be noted that those bits
  are informational only. In the default mode
  (when \emph{obey-ra-bits} is absent), the client will ask for
  configuration options that it has specified in the configuration
  file. With \emph{obey-ra-bits}, the client will wait till it
  receives the RA message and will act according to the received
  bits. The default is off (\emph{obey-ra-bits}
  missing). Enabling \emph{obey-ra-bits} implies \emph{inactive-mode}.

\item[experimental] -- (scope: global). Allows enabling experimental
features. There are some highly-experimental features present in
Dibbler. To make a clear statement about their experimental nature,
user is required to acknowledge that fact by putting this statement in
its config file. This statement may be present or absent. The default
is absent.

\item[addr-params] -- (scope: IA). Allows configuration of additional
sub-option conveyed in IAADDR. It supplements the usual information
about an address received from a server with prefix length. For
example, if client received address 2001:db8:1::abcd and addr-params
option contains 64, Dibbler client will configure prefix
2001:db8:1::/64 on the interface that was used to communicate with
server. This is experimental feature, not defined in any standard or
draft. Requires \emph{experimental} statement. See
Section \ref{feature-addr-params}.

\item[remote-autoconf] -- (scope: interface). Defines that remote
autoconfiguration should be performed on a given interface. This is
experimental feature, so it requires \emph{experimental}
statement. See Section \ref{feature-remote-autoconf}.

\item[ddns-protocol] -- (scope: global). Takes one string
parameter. Defines protocol that should be used during DNS Update
mechanism. Allowed values are \verb+tcp+, \verb+udp+ and \verb+any+.
Any means that UDP will be tried first and if it fails, update will be
retried over TCP. See Section \ref{feature-dns-update}.

\item[ddns-timeout] -- (scope: global). Takes one integer parameter
that specifies timeout in milliseconds. Defines how long client should
wait for DNS server response during DNS Update before declaring
update a failure. See Section \ref{feature-dns-update}.

\item[script] -- (scope: global). Takes one string parameter that
  specifies script name. When dibbler client receives some options it normally
  sets them up in the system on its own. However, besides of setting up all
  parameters directly, dibbler client can execute external
  script. See Section \ref{feature-script} for details.

\item[stateless] -- (scope: global). It may be present or missing. The
  default is missing. Defines that client should run in stateless
  mode. In this mode only configuration parameters are defined, not
  addresses or prefixes. It is mutually exclusive
  with \emph{ia}, \emph{ta} and \emph{pd}. See
  Section \ref{feature-stateless-stateful}. \emph{No-ia}, an alias to that
  command used to be supported, but due to misleading name its support
  was dropped in 0.8.1RC1.


\item[anonymous-inf-request] -- (scope: global). When running in a
  stateless mode, client does not ask for addresses or prefixes, but
  rather requests some general options. By default, it sends its
  client identifier (DUID) to the server. However, it is possible to
  omit this identifier, so the \msg{INF-REQUEST} messages will be
  anonymous. This global option causes client to act in such anonymous
  way.

 \item[inactive-mode] -- (scope: global). This parameter may be
   present or absent. The default is absent. Normally (with
   inactive-mode disabled) client tries to bind all interfaces defined
   in configuration file. If such attempt fails, client reports an
   error and gives up. In some cases it is possible that interface is
   not ready yet, e.g. WLAN interface did not complete association. It
   is possible to modify client behavior, so it will accept downed and
   not running interfaces. To do so, inactive-mode must be enabled. In
   this mode, client will accept inactive interfaces, will add them to
   inactive list and will periodically monitor its state. When the
   interface finally becomes available, client will try to configure
   it. See section \ref{feature-inactive-mode} for details.

 \item[insist-mode] -- (scope: global). Client can be instructed to
   obtain several configuration options, like DNS server configuration
   or domain name. It is possible that server will not provide all
   requested options. Older versions of the dibbler client had been
   very aggressive in such case. It tried very hard to obtain options
   that user specified in config file. To do so, it did
   send \msg{INF-REQUEST} to obtain such option. This behavior has
   changed. Right now when client does not receive all requested
   options, it will complain, but will take no action. To enable old
   behavior, so called insist-mode has been added. Insist-mode will
   also affect the way addresses are requested. If address was
   specified in config file, client will request it in \msg{REQUEST}
   message, rather than sending address offered by server
   in \msg{ADVERTISE} as it is typically done. See
   Section \ref{feature-insist-mode} for details.

 \item[skip-confirm] -- (scope: global). Support for \msg{CONFIRM} messages
   was added in 0.8.0RC1. With it, client may send \msg{CONFIRM} when link
   state change is detected (e.g. switching to possibly new WiFi
   access-point or replugging Ethernet cable). Client also sends
   \msg{CONFIRM} after restart if there are still valid leases found
   in locally stored databased. \emph{skip-confirm} will disable any
   actions that would result in \msg{CONFIRM} transmissions. In
   particular, link state will not be detected and client will ignore
   its previous address during startup.

%%% TODO: Add reconfigure accept here



 \item[duid-type] -- (scope: global). Takes one parameter. Allowed
   values are DUID-LLT, DUID-LL or DUID-EN. The default is DUID-LLT.
   This parameter defines, what type of DUID should be generated if
   there is no DUID already present. If there is a file containing
   DUID, this directive has no effect. DUID-LLT means that DUID will
   be based on link layer address as well as time. DUID-LL means that
   only link layer address will be used. The last value -- DUID-EN --
   Enterprise Number-based generation has a slightly different syntax:
 \begin{lstlisting}
   duid-type duid-en enterprise-number enterprise-id.
\end{lstlisting}
   For example: \verb+duid-type duid-en 1234 0x6789abcd+ means that
   enterprise number is set to 1234 and unique number from that
   company's pool is 67:89:ab:cd (hexadecimal value of arbitrary
   length). See section \ref{feature-duid-types} for details.

\item[option fqdn-s] -- (scope: global). Takes one boolean parameter
  and has the default value of 1. The S bit is used in FQDN option. It
  is used to negotiate, which side (server or client) wants to perform
  DNS Update procedure. See \cite{rfc4704} for details. In general, if
  you don't know that this option does, you don't want to modify this.

\item[option fqdn] -- (scope: interface). Takes optional domain
   name as parameter. This option instructs client to send FQDN
   option. This option has 2 purposes. The first one is to negotiated
   or request Fully Qualified Domain Name for this client. The second
   one is to negotiate, who (client or server) should perform DNS
   Update. If optional parameter is specified, it will be sent in the
   FQDN option. Otherwise FQDN will be sent with empty name. This
   option is defined in \cite{rfc4704}. See
   Section \ref{feature-dns-update} for details.

 \item[rapid-commit] -- (scope: interface). Takes one boolean
   parameter. The default is 0. This option allows rapid commit
   procedure to be performed. Note that enabling rapid-commit on the
   client side is not enough. server must be configured to allow rapid
   commit, too.

 \item[unicast] -- (scope: interface). Takes one boolean
  parameter. The default value is 0. This option specifies if client
  should request unicast communication from the server. If server is
  configured to allow it, it will add unicast option to its
  replies. It will allow client to communicate with server via unicast
  addresses instead of usual multicast.

 \item[preferred-servers] -- (scope: interface). Takes list of
  addresses or list of DUIDs. The default value is empty.  This list
  defines, which servers are preferred. When client sends \msg{SOLICIT}
  message, all servers available in the local network will
  respond. When client receives multiple \msg{ADVERTISE} messages, it
  will choose those sent by servers mentioned on the preferred-server
  list. Otherwise the best server will be chosen. Note that this
  parameter used to be spelled differently (single R). Old spelling
  is still supported, but is deprecated and will be removed soon.

 \item[reject-servers] -- (scope: interface). Takes list of addresses
  or list of DUIDs. The default value is empty. This list defines
  which server must be ignored. It has opposite meaning to the
  prefered-servers list. Servers mentioned here will never be chosen.

 \item[vendor-spec] -- (scope: interface). This option allow
   requesting for a vendor specific configuration option or
   options. Although there are no vendor-specific (i.e. specific for
   Dibbler) parameters, it can be used to test some other DHCPv6
   server implementations. It takes coma separated list following
   tokens: type (integer), type(integer) -- enterprise(number). It
   allows definition of just vendor-specific option and/or
   vendor-specific option with specific enterprise. See feature
   description in Section \ref{feature-vendor-spec}.

 \item[T1] -- (scope: IA or PD) Takes one parameter that specifies T1
   hint sent to a server. The default value is $2^{32}-1$ and is
   expressed in seconds. T1 value assigned by server defines after
   what time client should start renew process. This is only a hint
   for the server. Actual value will be provided by the server.

\item[T2] -- (scope: IA or PD). Takes one parameter that specifies T2 hint
   sent to a server. The default value is default:$2^{32}-1$ and is
   expressed in seconds. This value defines hint for T2. T2 assigned
   by server defined after what time client will start emergency
   rebind procedure if renew process fails. This is only a hint for
   the server. Actual value will be provided by the server.

\item[valid-lifetime] -- (scope:address or prefix). Takes one integer
   parameter that defined requested valid lifetime for address or
   prefix. The default value is $2^{32}-1$. This parameter is
   expressed in seconds. This parameter defines valid lifetime of an
   address. It will be used as a hint for a server, when the client
   sends \msg{REQUEST} messages.

\item[preferred-lifetime] -- (scope:address or prefix).
   type: integer, default:$2^{32}-1$) This parameter defines prefered
   lifetime of an address. It will be used as a hint for a server,
   when there client sends \msg{REQUEST} messages. Note that this
  parameter used to be spelled differently (single R). Old spelling
  is still supported, but is deprecated and will be removed soon.

 \item[option dns-server] -- (scope: interface). Takes optional
   address list as parameter. This option conveys information about
   DNS servers available. After retriving this information, client
   will be able to resolve domain names into IP (both IPv4 and IPv6)
   addresses. If address list is not specified, client will just
   request dns-server option from a server. If it is specified, listed
   addresses will be sent to server as hints. Defined in \cite{rfc3596}.

 \item[option domain] -- (scope: interface). Takes optional domain
   list as parameter. This option is used for retriving domain or
   domains names, which the client is connected in. For example, if
   client's hostname is \verb+alice.mylab.example.com+ and it wants to
   contact \verb+bob.mylab.example.com+ it can simply refer to it
   as \verb+bob+. Without domain name configured, it would have to use
   full domain name. If optional domain list if defined, it will be
   sent to server as a hint. Defined in \cite{rfc3596}.

 \item[option ntp-server] -- (scope: interface). Takes optional address
   list as parameter. This option defines information about available NTP
   servers. Network Time Protocol \cite{rfc2030} is a protocol used
   for time synchronisation, so all hosts in the network has the same
   proper time set. If address list is specified, it will be sent to
   server as a hint. Defined in \cite{rfc4075}.

 \item[option time-zone] -- (scope: interface). Optionally takes one
   string parameter that specifies requested timezone. It is possible
   to retrieve timezone from the server. If client is interested in
   this information, it should ask for this option. Note that this
   option is considered obsolete as it is mentioned in draft version
   only \cite{draft-timezone}. Work on this draft seems to be
   abandoned as similar functionality is provided in now
   standard \cite{rfc4075}. Unfortunately, it is not supported in
   Dibbler yet.

 \item[option sip-server] -- (scope: interface). Takes optional
   address list as parameter. Session Initiation Protocol
   (SIP) \cite{rfc3263} is a control protocol for creating, modifying,
   and terminating sessions with one or more participants. These
   sessions include Internet telephone calls, multimedia distribution,
   and multimedia conferences, including VoIP phones. If address list
   is specified, it will be sent as a hint for a server. Format of
   this option is defined in \cite{rfc3319}.

 \item[option sip-domain] -- (scope: interface). Takes optional list
   of domains. It is possible to define domain names for Session
   Initiation Protocol \cite{rfc3263}. Configuration of this parameter
   will ease usage of domain names in the SIP protocol. If domain list
   is specified, it will be sent as a hint for a server. Format of
   this option is defined in \cite{rfc3319}.

 \item[option nis-server] -- (scope: interface). Takes optional
   address list as parameter. Network Information Service (NIS) is a
   Unix-based system designed to use common login and user information
   on multiple systems, e.g. universities, where students can log on
   to ther accounts from any host. To use this functionality, a host
   needs information about NIS server's address. This can be retrieved
   with this option. If address list is specified, it will be sent as
   a hint for a server. Its format is defined in \cite{rfc3898}.

 \item[option nis-domain] -- (scope: interface). Takes optional list
   of domains. Network Information Service (NIS) can albo specify
   domain names. It can be configured with this option. If domain list
   is specified, it will be sent as a hint for a server. It is defined
   in \cite{rfc3898}.

 \item[option nis+-server] -- (scope: interface). Takes optional
   address list as parameter. Network Information Service Plus (NIS+)
   is an improved version of the NIS protocol. If address list is
   specified, it will be sent as a hint for a server. This option is
   defined in \cite{rfc3898}.

 \item[option nis+-domain] -- (scope: interface). Takes optional list
   of domains. Similar to nis-domain, it defines domains for NIS+. If
   domain list is specified, it will be sent as a hint for a
   server. This option is defined in \cite{rfc3898}.

 \item[option lifetime] -- (scope: interface). This statement can be
   present or missing. The default is missing.
   Base spec of the DHCPv6 protocol does offers way of refreshing
   addresses only, but not the options. Lifetime defines, how often
   client would like to renew all its options. By default client will
   not send such option, but it will accept it and act accordingly if
   the server sends it on its own. Format of this option is defined
   in \cite{rfc4242}.

 \item[option aftr] -- (scope: interface). In networks that deploy
   Dual-Stack Lite architecture \cite{rfc6333}, client (B4) needs to
   configure DS-Lite tunnel. Client may obtain information about AFTR
   (a remote tunnel endpoint). This option conveys fully qualified
   domain name. This statement instructs client to request such
   option. It is defined in defined in \cite{rfc6334}.

\item[option] -- (scope: interface). There are number of options
   supported by Dibbler. There may be cases, however, when user wants
   to specify its own options. Several syntaxes are supported:
\begin{lstlisting}
option number - hexstring
option number address-keyword address
option number address-list
option number string-keyword string
option number address-keyword request-keyword
option number string request-keyword
option number address-list request-keyword
\end{lstlisting}
where number designates option number, address-keyword is word ``address'',
address is an IPv6 address, address-list is coma separated list of
IPv6 addresses, string-keyword is a word ``string'' and
request-keyword is a word ``request''. See Section \ref{feature-custom-options}.

\item[auth-protocol] -- (scope: global, type: string, default:
  none). This is a crucial parameter that specifies which
  authorization/authentication protocol is used. Allowed values are:
  \texttt{none}, \texttt{delayed}, \texttt{reconfigure-key} and
  \texttt{dibbler}. See section \ref{feature-auth} for details.

 \item[auth-methods] -- (scope: global). Takes coma separated
   list of accepted authentication methods methods that client will
   accept from server. If this list is empty, any method will be
   accepted. The first method on the list is the default one. Possible values
   are:  \texttt{none}, \texttt{digest-plain}, \texttt{digest-hmac-md5},
   \texttt{digest-hmac-sha1}, \texttt{digest-hmac-sha224},
   \texttt{digest-hmac-sha256}, \texttt{digest-hmac-sha384},
   and \texttt{digest-hmac-sha512}. 

\item[auth-replay] -- (scope: global, type: string, default:
  none). Specifies which replay detection methods are
  supported. Currently two values are implemented: \texttt{none} and
  \texttt{monotonic}.

\item[auth-required] -- (scope: global, type: boolean, default:
  0). This parameter specifies if the client is required to
  authenticate itself. When set to 0, any client authentication
  failures (invalid signature or lack of \opt{AUTH} option) will
  result in a warning only. When set to 1, such messages will be
  dropped.

 \item[route] -- (scope: interface). Takes one boolean parameter that
 defines if routing information should requested or not. The default
 value is false. See Section \ref{feature-routing}.

\end{description}

After receiving options values from a server, client stores values of
those options in separate files in the working directory
(\verb+/var/lib/dibbler+ in POSIX systems (Linux, Mac OS X and BSD)
and current directory in Windows). File names start with the option
word, e.g. \verb+option-dns-server+. Dibbler client can also call user
defined script after parameters are assigned or removed. Dibbler
client also sets DNS servers and domain names on its own on most
systems.

\subsection{Client Configuration Examples}
This subsection contains various examples of the most popular
configurations. Several additional examples are provided with the source
code. Please download it and look at \verb+*.conf+ files.

\subsubsection{Example 1: Default}
In the most simple case, client configuration file can be empty. Client will try to
assign one address for every interface present in the system, except
interfaces, which are:
\begin{itemize}
\item down (flag UP not set)
\item loopback (flag LOOPBACK set)
\item not running (flag RUNNING not set)
\item not multicast capable (flag MULTICAST not set)
\item have link-layer address less than 6 bytes long (this requirement
      should skip all tunnels and virtual interfaces)
\end{itemize}

If you must use DHCPv6 on one of such interfaces (which is not
recommended and such attempt probably will fail), you must explicitly
specify this interface in the configuration file.

\subsubsection{Example 2: DNS}
Configuration mentioned in previous subsection is a minimal one and in a
real life will be used rarely. The most common usage of the DHCPv6
protocol is to request for an address and DNS configuration. Client
configuration file achieving those goals is presented below:
\begin{lstlisting}
# client.conf
log-mode short
log-level 7
iface eth0 {
  ia
  option dns-server
}
\end{lstlisting}

\subsubsection{Example 3: Timeouts and specific address}
\label{example-specific-addrs}
Automatic configuration is being driven by several timers, which define,
what action should be performed at various intervals. Since all
values are provided by the server, client can only define values, which
will be sent to a server as hints. Server might take them into
consideration, but might also ignore them
completely. Following example shows how to ask for a specific address
and provide hints for a server. Client would like to get 2000::1:2:3
address, it would like to renew addresses once in 30 minutes (T1 timer
is set to 1800 seconds). Client also would like to have address, which
is prefered for an hour and is valid for 2 hours.

Note: The format has changed in 1.0.0RC2.

\begin{lstlisting}
# client.conf
log-mode short
log-level 7
iface eth0 {
  T1 1800
  T2 2000
  prefered-lifetime 3600
  valid-lifetime 7200
  ia {
    address 2000::1:2:3
  }
}
\end{lstlisting}

There are multiple ways in which addresses can be requested in
ia. This syntax was implemented more for completeness, rather than
having practical utility. It is mentioned here for reference.

\begin{lstlisting}
# client.conf

iface eth0 {
   T1 1800
   T2 2700

   ia // Send just an empty IA

   ia { // Send an IA with one (any, ::) address
       address
   }

   ia { // Send an IA with five (any, ::) addresses
       address 5
   }

   ia { // Send an IA with address 2001:db8::1
       address 2001:db8::1
   }

   ia { // Send an IA with one (any, ::) address with specific parameters
       address {
          preferred-lifetime 3600
       }
   }

   ia { // Send an IA with five (any, ::) addresses with specific parameters
       address 5 {
          preferred-lifetime 3600
       }
   }

   ia { // Send an IA with address 2001:db8::1
       address 2001:db8::1 {
          preferred-lifetime 3600
       }
   }
}
\end{lstlisting}

\subsubsection{Example 4: More than one address}

Another example: client would like to obtain 2 addresses on
wifi0 interface. They are necessary since this particular interface name
contains spaces. It is possible to do this in two ways. First is to
sent 2 Identity Associations (IA for short). Identity Association is a
nice name for a addresses container. This appears to be a most common
way of telling server that this client is interested in more than one
address.

\begin{lstlisting}
# client.conf
log-mode short
log-level 5
iface wifi0 {
  ia
  ia
}
\end{lstlisting}

Another way it to send one IA, but include two address hints in
it. Server may take them into consideration (dibbler server does), but
some other DHCPv6 implementations may ignore those hints.

\begin{lstlisting}
# client.conf
log-mode short
log-level 5
iface wifi0 {
  ia {
     address
     address
  }
}
\end{lstlisting}

\subsubsection{Example 5: Quick configuration using Rapid-commit}
Rapid-commit is a shortened exchange with server. It consists of only
two messages, instead of the usual four. It is worth to know that both sides (client
and server) must also support rapid-commit to use this fast
configuration.

\begin{lstlisting}
# client.conf
iface eth1 {
  rapid-commit yes
  ia
  option dns-server
}
\end{lstlisting}

\subsubsection{Example 6: Stateless mode}
Client can be configured to work in a stateless mode. It means that it
will obtain only some configuration parameters, but no
addresses. Let's assume we want all the details stored in a log file and
we want to obtain all possible configuration parameters. Here is a
configuration file:

\begin{lstlisting}
# client.conf
log-level 8
log-mode full
stateless
iface eth0
{
  option dns-server
  option domain
  option ntp-server
  option time-zone
  option sip-server
  option sip-domain
  option nis-server
  option nis-domain
  option nis+-server
  option nis+-domain
}
\end{lstlisting}

\subsubsection{Example 7: Dynamic DNS (FQDN)}
\label{example-client-fqdn}
Dibbler client is able to request fully qualified domain name,
i.e. name, which is fully resolvable using DNS. After receiving such
name, it can perform DNS Update procedure. Client can ask for any
name, without any preferrence. Here is an example how to configure
client to perform such task:
\begin{lstlisting}
# client.conf

# Set protocol to one of the following values: udp, tcp, any
ddns-protocol udp

# Sets DDNS Update timeout (in ms)
ddns-timeout 800

# uncomment following line to force S bit to 0
# option fqdn-s 0
log-level 7

iface eth0 {
# ask for one address
    ia

# ask for options
   option dns-server
   option domain
   option fqdn

# ask for fully qualified domain name (any name will do)
  option fqdn
# you can also provide hint for the server regarding preferred name
#  option fqdn dexter.example.org

}
\end{lstlisting}

In this case, client will mention that it is interested in FQDN by
using Option Request and empty FQDN option, as specified in
\cite{rfc4704}. Server upon receiving such request (if it is
configured to support it), will provide FQDN option containing domain
name. Depending on the server's configuration, all DNS Updates will be
performed by the server, forward will be performed by client and reverse
by the server, or only forward will be done by a client.

It is also possible for client to provide its name as a hint for
server. Server might take it into consideration when it will choose a
name for this client. To send specific hostname, additional parameter
(a string with a fully qualified domain name) should be specified.

Two additional parameters were introduced in Dibbler
0.8.1. \verb+ddns-protocol+ specifies protocol that should be used for
communication with DNS server.  Allowed values
are \verb+udp+, \verb+tcp+ or \verb+any+. ``Any'' will try to use UDP
and if that fails, it will revert to TCP. Second parameter
is \verb+ddns-timeout+ that specifies maximum time allowed for DNS
server to respond before assuming communication failure. It is
specified in milliseconds.

Note that to successfully perform DNS Update, address must be assigned
and dns server address must be known. Therefoe  ``ia'' and ``option
dns-server'' are required for ``option fqdn'' to work properly. Also if
DHCPv6 server provides more than one DNS server address, update will
be attempted only for the first address on the list.

It is also possible to force S bit in the FQDN option to 0 or 1. See
\cite{rfc4704} for details regarding its meaning.

\subsubsection{Example 8: Interface indexes}
Usually, interface names are referred to by names, e.g. eth0 or Local
Area Connection. Every system also provides unique number associated
with each infterface, usually called ifindex or interface index. It is
possible to read the number using \verb+ip l+ command (Linux) or
\verb+ipv6 ifx+. Below is an example, which demonstrate how to use
interface indexes:

\begin{lstlisting}
# client.conf
log-mode short
log-level 5
iface 5 {
  ia
}
\end{lstlisting}

\subsubsection{Example 9: Vendor-specific options}
\label{example-client-vendor-spec}
It is possible to configure dibbler-client to ask for a vendor specific
options. Although there are no dibbler-specific features to configure,
it is possible to use this option to test other server
implementations. This option will rather be used by network engineers
and power network admins, rather than normal end users.

There are 3 ways to define, how dibbler-client can request
vendor-specific options. First choice: It can just ask for this option (only
\opt{option request option} will be sent). Second choice: it can ask for
vendor-spec option by adding such option with enterprise number set, but
no actual data. Third choice: send this option and include both
enterprise number and actual data. In the following configuration file
example, uncomment appropriate line to obtain desired bahavior:

\begin{lstlisting}
# client.conf
log-level 8
iface eth0 {
# ask for address
    ia

# uncomment only one of the following lines:
   option vendor-spec
#   option vendor-spec 1234
#   option vendor-spec 1234 5678

# To ask for multiple vendor-spec options, uncomment:
#   option vendor-spec 123,456
}
\end{lstlisting}

Although that is almost never needed, it is possible to configure
client to request multiple vendor-specific options at the same
time. That feature is mainly used as a test tool for the server. To
use it, uncomment last line in the example above.

\subsubsection{Example 10: Unicast communication}

Client would like to obtain an address on ,,Local Area Connection''
interface. Note quotation marks around
interface name. They are necessary since this particular interface name
contains spaces. Client also would like to accept Unicast
communication if server supports it. User wants all information
to be logged via Linux syslog daemon. Take note that you won't be
able see to what Dibbler is doing with such low log-level. (Usually
log-level should be set to 7, which is also a default value).

\begin{lstlisting}
# client.conf
log-mode syslog
log-level 5
iface "Local Area Connection" {
  unicast yes
  ia
  ia
}
\end{lstlisting}

\subsubsection{Example 11: Prefix delegation}
\label{example-client-prefix}
From the client's point of view, configuration is quite simple. It is
required to specify that this client is interested in prefix
delegation. See section \ref{feature-prefix} for background
information related to prefix delegation and sections
\ref{example-server-prefix} and \ref{example-server-prefixes} for
details about server configuration. To ask for prefix delegation,
emph{prefix-delegation} (or \emph{pd}) should be used.

\begin{lstlisting}
# client.conf
iface "eth0" {
  ia  // ask for address
  pd  // ask for prefix
}
\end{lstlisting}

It is possible to define additional parameters for a prefix:

\begin{lstlisting}
# client.conf
iface eth0 {
  pd {
    t1 1000
    t2 2000
  }
}
\end{lstlisting}

Client (requesting router in PD nomenclature) receives prefix from
upstream router and tries to auto-select downstream interfaces. It
tries to use interfaces that are up, running, multicast-capable, with
MAC address at least 6 bytes long and were not used to obtain
prefix. If this algorithm does not work in your case (e.g. because you
want to use prefixes on other interfaces or you want some interfaces
to be skipped), it is possible to explicitly enumerate downstream
interfaces using \emph{downlink-prefix-ifaces}:

\begin{lstlisting}
# client.conf

# received prefix will be split among following interfaces
downlink-prefix-ifaces eth1, eth5

# Ask for prefix over eth0
iface eth0 {
    pd
}
\end{lstlisting}

If you do not want Dibbler to split the prefixes automatically, it is
possible to do so by specifying \emph{"none"} as the interface
name. Note that this will render PD mechanism useless, unless you also
use a script and do the delegated prefix processing on your own.

\begin{lstlisting}
# client.conf

# Dibbler client should not split received prefixes on its own
downlink-prefix-ifaces "none"

# You need to provide your own script to handle prefixes
script "/var/lib/dibbler/client-pd-split.sh"

# Ask for prefix over eth0
iface eth0 {
    pd
}
\end{lstlisting}

Prefix hints can be specified in the similar way as addresses (see
\ref{example-specific-addrs}, except that multiple prefixes syntax
is not supported.

\begin{lstlisting}
# client.conf
log-level 8

iface eth0 {
   T1 1800
   T2 2700

   pd // Send just an empty PD

   pd { // Send a PD with one (any, ::/0) prefix
       prefix
   }

   pd { // Send an PD with a specific prefix
       prefix 2001:db8::1 / 64
   }

   pd { // Send an PD with one (any, ::/0) prefix with specific parameters
       prefix {
          preferred-lifetime 3600
       }
   }

   pd { // Send an PD with a specific prefix and specific parameters
       prefix 2001:db8::1 /64 {
          preferred-lifetime 3600
       }
   }
}
\end{lstlisting}


\subsubsection{Example 12: Insist mode}
\label{example-client-insistmode}
During normal operation, when client asks for an option, but does not
receive it from the server, it complain, but takes no action. To force
client to insist (i.e. ask over and over again), so called insist mode
has been introduced. See section \ref{feature-insist-mode} for
extended explanation.

\begin{lstlisting}
insist-mode
iface "eth0" {
   ia
   option dns-server
   option domain
   option ntp-server
}
\end{lstlisting}

\subsubsection{Example 13: Inactive mode}
\label{example-client-inactivemode}
Usually client starts when network interfaces are
operational. Normally downed or nonexisting interfaces mentioned in
the configuration file are considered misconfiguration and client
refuses to start. However, sometimes that is not the case, e.g. still
waiting to be associated wireless interfaces. To allow operation in
such circumstances, inactive mode has been added. See
\ref{feature-inactive-mode} for detailed explanation.
interfaces are spec

\begin{lstlisting}
inactive-mode
iface "eth0" {
  ia
}
\end{lstlisting}

\subsubsection{Example 14: Dibbler Authentication}
\label{example-client-auth}
Authentication is enabled. Client will accept HMAC-SHA-512, HMAC-MD5
and HMAC-SHA-256 as an authentication method.

\begin{lstlisting}
# client.conf

log-mode short
log-level 7

auth-protocol dibbler
auth-replay monotonic
auth-required 1
auth-methods digest-hmac-sha512, digest-hmac-md5, digest-hmac-sha256

iface eth0 {
}
\end{lstlisting}

\subsubsection{Example 15: Skip Confirm}
\label{example-client-confirm}
Client detects if previous client instance was not shutdown properly
(due to power outage, client crash or similar event). In such case, it
reads existing address database and checks if assigned addresses may
still be valid. If that is so, it tries to confirm those addresses by
using \msg{CONFIRM} message.

If user don't want \msg{CONFIRM} message to be send and client should
start ''from scratch'' every time, it is possible to disable confirm
support.

\begin{lstlisting}
# client.conf

log-mode short
log-level 7
skip-confirm

iface eth0 {
  ia
}
\end{lstlisting}

\subsubsection{Example 15: User-defined IAID}
\label{example-client-iaid}
Sometimes it is useful to define specific IAID  identifiers. That is
rather uncommon, but possible. This technique can be used for both
addresses (IA\_NA options) and prefixes (IA\_PD options).

\begin{lstlisting}
# client.conf

iface "eth0" {
  ia 123
  option dns-server
  option domain
}
\end{lstlisting}

\subsubsection{Example 16: DS-Lite tunnel (AFTR)}
\label{example-server-ds-lite}
Server may provide information about AFTR (a Dual Stack Lite tunnel
endpoint) to the clients, as specified in \cite{rfc6334}.

\begin{lstlisting}
iface "eth0" {
  ia
  option aftr # request name of the remote DS-Lite tunnel endpoint
}
\end{lstlisting}

\subsubsection{Example 17: Custom options}
Client is able to ask for custom options, that are not supported by
default. Following config file allows client to ask for many
options. Also, see Section \ref{feature-custom-options} for
extended explanation. Note that the syntax changed slightly
after Dibbler 0.8.3 was released.

\begin{lstlisting}
#client.conf
iface "eth0" {
  ia

  # This will send specified option value  
  option 145 hex 01:02:a3:b4:c5:dd:ea
  option 146 address 2001:db8:1::dead:beef
  option 147 address-list 2001:db8:1::aaaa,2001:db8:1::bbbb
  option 148 string "secretlair.example.org"

  # This will request specific options and interpret responses
  option 149 hex
  option 150 address
  option 151 address-list
  option 152 string
\end{lstlisting}

\subsubsection{Example 18: Remote Autoconfiguration}
Client is able to use experimental extension to ask for configuration
remotely. See Section \ref{feature-remote-autoconf} for details.

\begin{lstlisting}
log-mode short
log-level 8
experimental
remote-autoconf

iface "eth0" {
  ia
  unicast 1
  option dns-server
  option domain
  option nis-server
  option nis-domain
  option nis+-server
  option nis+-domain
  option time-zone
  option lifetime
}
\end{lstlisting}
